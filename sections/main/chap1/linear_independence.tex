\subsection{Linear Independence}
\begin{definition}[Linear Independence]
	We have the following definitions of linear independence:
	\begin{enumerate}	
		\item In any vector space, a finite set of vectors $\vv_1, \dots, \vv_n$ is said to be linearly indepedent if \textup{\textbf{\underline{NONE of its elements} is a linear combination of the others from the set}}. Otherwise, the set is linearly dependent.
		\item A finite set of vectors $\vv_1, \dots, \vv_n$ is linearly independent if and only if:
		\begin{align*}
			\alpha_1\vv_1 + \dots + \alpha_n\vv_n = 0 \implies \alpha_i = 0, \quad \forall 1\le i\le n.	
		\end{align*} 

		\item A finite set of vectors $S=\{\vv_1, \dots, \vv_n\}$ is linearly independent if and only if:
		\begin{align*}
			\forall 1 \le k \le n: \vv_k \notin \mathrm{span}(S\setminus\{\vv_k\}),
		\end{align*} 

		\noindent In other words, every vector in $S$ is not spanned by the remaining vectors of $S$.
	
		\item An \underline{infinite} set of vectors $V=\{\vv_k\}_{k=1}^\infty$ is linearly indepedent if every finite subset of $V$ is linearly independent.
	\end{enumerate} 
\end{definition} 

\begin{definition}[Relation between Linear Independence and Span]
	From propositions \ref{prop:add_vec_to_span}, \ref{prop:remove_vec_from_span} and corollaries \ref{coro:add_ind_vec_to_span}, \ref{coro:remove_vec_from_lin_ind_sets}, we have the following properties that relate linear independence to spanning sets:
\end{definition} 

\begin{table}[ht!]
    \begin{center}
    \begin{tabular}{@{}lcc@{}}
	\toprule
	                               & \multicolumn{1}{c}{\textbf{$\mathrm{span}(S)=\mathrm{span}(\tilde S)$}} & \multicolumn{1}{c}{\textbf{$\mathrm{span}(S)\ne\mathrm{span}(\tilde S)$}} \\ \midrule
	\textbf{Add $\vv$ to $S$}      & $\vv\in\mathrm{span}(S)$                                                & $\vv\notin\mathrm{span}(S)$                                               \\
	\textbf{Remove $\vv$ from $S$} & $\vv\in\mathrm{span}(\tilde S)$                                         & $\vv\notin\mathrm{span}(\tilde S)$                                        \\ \bottomrule
	\end{tabular}
    \caption{Conditions for $\mathrm{span}(\tilde S) = \mathrm{span}(S)$ where $\tilde S$ is resulted from adding $\vv$ to or removing $\vv$ from $S$. Note that for a vector $\vu$ and a (finite) set $V$, when we write $\vu\notin\mathrm{span}(V)$, it is the same as saying ``$\vu$ is independent of all vectors in $V$''.}
    \end{center}
\end{table}

\begin{definition}[Properties of Linear Independence]
	Let $V$ be a vector space (over a field $\field$) and $S\subseteq V$ be a finite subset of $V$. Then, we have:
	\begin{enumerate}
		\item $S$ is linearly dependent if and only if there are distinct $\vv_0, \dots, \vv_n\in S$ and $\lambda_1, \dots, \lambda_n\in S$ (not all zeros) such that $\vv_0 = \sum_{i=1}^n\lambda_i\vv_i$.
		\item Let $S_1 \subseteq S \subseteq S_2$, then:
		\begin{itemize}
			\item $S$ is linearly independent $\implies$ $S_1$ is linearly independent.
			\item $S$ is linearly dependent $\implies$ $S_2$ is linearly dependent.	
		\end{itemize} 

		\item There always exists $\tilde S \subseteq S$ such that $\tilde S$ is linearly independent and $\mathrm{span}(S) = \mathrm{span}(\tilde S)$.
	\end{enumerate} 
\end{definition} 

\begin{example}[Finding the Minimum Spanning Set]
	Suppose that we have a set of vectors $S=\{\vv_1, \vv_2, \vv_3 \}$ where each $\vv_i\in\R^3$. Then, we can find the subset $S_1\subseteq S$ by:
	\begin{enumerate}[label=(\roman*)]
		\item Write $S$ as a matrix $A$ where the $i^{th}$ column is $\vv_i$.
		\item Write $A$ in reduced row echelon form: $A\to A_\mathrm{rref}$.
	\end{enumerate} 
\end{example} 

