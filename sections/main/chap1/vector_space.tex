\subsection{Definition of Vector Space}
\begin{definition}[Vector Space]
	A \textbf{vector space} (over a field $\field$) consists of a set $V$ with two operations $``+"$ and $``\cdot"$ subject to the conditions that for all $\vec{v}, \vec{w}, \vec{u}\in V$ and scalars $r, s\in\field$:
	\begin{enumerate}
		\item \textbf{Closure under}:
		\begin{itemize}
			\item Vector addition: $\vec{v} + \vec{w} \in V$.
			\item Scalar multiplication: $r\cdot\vec{v} \in V$.
		\end{itemize}

		\item \textbf{Properties of vector addition}:
		\begin{itemize}
			\item Commutativity: $\vec{v} + \vec{w} = \vec{w} + \vec{v}$.
			\item Associativity: $(\vec{v} + \vec{w}) + \vec{u} = \vec{v} + (\vec{w} + \vec{u})$.
		\end{itemize} 

		\item \textbf{Properties of scalar multiplication}:
		\begin{itemize}
			\item Distributivity over scalar addition: $(r+s)\cdot\vec{v} = r\cdot\vec{v} + s\cdot\vec{v}$.	
			\item Distributivity over vector addition: $r\cdot(\vec{v} + \vec{w}) = r\cdot\vec{v} + r\cdot\vec{w}$.
		\end{itemize} 

		\item \textbf{Inverse elements}:
		\begin{itemize}
			\item Additive inverse: $\forall \vec{v}\in V, \exists -\vec{v}\in V: \vec{v} + (-\vec{v}) = \vec{0}$.	
		\end{itemize} 

		\item \textbf{Identity elements}:
		\begin{itemize}
			\item Additive identity: $\exists\vec{0} \in V: \vec{0} + \vec{v} = \vec{v}, \quad \forall \vec{v}\in V$.
			\item Multiplicative identity: $\exists \bm{1}\in\field: \bm{1}\cdot\vec{v} = \vec{v},\quad \forall \vec{v} \in V$.	
		\end{itemize} 
	\end{enumerate} 

	\noindent For brevity, we will denote vectors as bold face letters instead of overhead arrows after this definition. For example, $\vu, \vv$ and $\vw$.
\end{definition} 

\begin{remark}[``Over a field'']
	When we use the phrase ``a vector space over a field $\field$'', this means that the scalars that we use will be taken from the field $\field$. It does not mean that our vector space consists of $\field$-valued vectors. For example, the following vector space:
	\begin{align*}
		\mathrm{L} = \bigCurl{(x, \alpha x) : x\in\mathbb{C}, \alpha\in\R}	
	\end{align*}
	is a vector space over $\R$ (scalar multiplications are done with real-valued scalars) even though the vectors are complex-valued.
\end{remark} 

\begin{remark}[Trivial Space]
	A vector space with one element is called a \textbf{trivial space}.	
\end{remark} 

\begin{example}[A simple example]
	The following is a vector space over $\R$:
	\begin{align*}
		\mathrm{L} = \bigCurl{
			\begin{pmatrix}
				x & y
			\end{pmatrix}^\top: y = 3x 
		}.
	\end{align*}

	\noindent This is easy to verify. Let us go through each condition one by one. Let the vectors $\vu_1, \vu_2, \vu_3 \in \mathrm{L}$ defined as follows: 
	\begin{align*}
		\vu_1 = \begin{pmatrix}
			x_1 \\ 3x_1	
		\end{pmatrix}, \quad
		\vu_2 = \begin{pmatrix}
			x_2 \\ 3x_2	
		\end{pmatrix}, \quad 
		\vu_3 = \begin{pmatrix}
			x_3 \\ 3x_3	
		\end{pmatrix}..
	\end{align*} 

	\noindent All the axioms of a vector space are satisfied. Let $\alpha,\beta\in\R$, we have:
	\begin{enumerate}
		\item \emph{Closure under vector addition}: $\vu_1 + \vu_2 = \begin{pmatrix} x_1 \\ 3x_1 \end{pmatrix} + \begin{pmatrix} x_2 \\ 3x_2 \end{pmatrix} = \begin{pmatrix} x_1 + x_2 \\ 3(x_1 + x_2) \end{pmatrix} \in \mathrm{L}$.

		\item \emph{Closure under scalar multiplication}: $\alpha\vu_1 = \alpha \begin{pmatrix} x_1 \\ 3x_1 \end{pmatrix} = \begin{pmatrix} \alpha x_1 \\ 3\alpha x_1 \end{pmatrix} \in \mathrm{L}$. 

		\item \emph{Additive commutativity}: $\vu_1 + \vu_2 = \begin{pmatrix} x_1 + x_2 \\ 3x_1 + 3x_2 \end{pmatrix} = \begin{pmatrix} x_2 + x_1 \\ 3x_2 + 3x_1 \end{pmatrix} = \vu_2 + \vu_1$.

		\item \emph{Additive associativity}: $(\vu_1 + \vu_2) + \vu_3 = \begin{pmatrix} (x_1 + x_2) + x_3 \\ 3(x_1 + x_2) + 3x_3\end{pmatrix}=\begin{pmatrix} x_1 + (x_2 + x_3) \\ 3x_1 + 3(x_2 + x_3)\end{pmatrix} = \vu_1 + (\vu_2 + \vu_3)$.

		\item $\dots$ (We can easily verify other axioms as well).
	\end{enumerate} 
\end{example} 

\begin{example}[Polynomials of degree 3]
	Consider the following set of real-coefficients polynomials with degree of at most 3:
	\begin{align*}
		\mathcal{P}_3 &= \bigCurl{
			a_0 + a_1x + a_2x^2 + a_3x^3 \Big| a_0, a_1, a_2, a_3\in\R
		}.
	\end{align*} 

	\noindent Then, $\mathcal{P}_3$ is a vector space over $\R$ under the following operations:
	\begin{align*}
		(a_0 + a_1x + a_2x^2 + a_3x^3) &+ (b_0 + b_1x + b_2x^2 + b_3x^3) \\
			&= (a_0 + b_0) + (a_1 + b_1)x + (a_2 + b_2)x^2 + (a_3 + b_3)x^3, \\
		\alpha\cdot(a_0 + a_1x + a_2x^2 + a_3x^3) &= (\alpha a_0) + (\alpha a_1) x + (\alpha a_2) x^2 + (\alpha a_3) x^3.
	\end{align*} 

	\noindent We can think of $\mathcal{P}_3$ as being ``the same'' as the vector space $\R^4$. For every set of real coefficients $a_0, \dots, a_3$, we have the following correspondence:
	\begin{align*}
		a_0 + a_1x + a_2x^2 + a_3x^3 \text{ corresponds to } \begin{pmatrix}
			a_0 \\ a_1 \\ a_2 \\ a_3	
		\end{pmatrix}.
	\end{align*}

	\noindent Similarly, for any $n\ge 1$, $\mathcal{P}_n$ is also a vector space over $\R$.
\end{example} 