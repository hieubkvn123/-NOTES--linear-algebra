\subsection{Definition of Vector Space}
\begin{definition}[Vector Space]
	A \textbf{vector space} (over a field $\field$) consists of a set $V$ with two operations $``+"$ and $``\cdot"$ subject to the conditions that for all $\vec{v}, \vec{w}, \vec{u}\in V$ and scalars $r, s\in\field$:
	\begin{enumerate}
		\item \textbf{Closure under}:
		\begin{itemize}
			\item Vector addition: $\vec{v} + \vec{w} \in V$.
			\item Scalar multiplication: $r\cdot\vec{v} \in V$.
		\end{itemize}

		\item \textbf{Properties of vector addition}:
		\begin{itemize}
			\item Commutativity: $\vec{v} + \vec{w} = \vec{w} + \vec{v}$.
			\item Associativity: $(\vec{v} + \vec{w}) + \vec{u} = \vec{v} + (\vec{w} + \vec{u})$.
		\end{itemize} 

		\item \textbf{Properties of scalar multiplication}:
		\begin{itemize}
			\item Distributivity over scalar addition: $(r+s)\cdot\vec{v} = r\cdot\vec{v} + s\cdot\vec{v}$.	
			\item Distributivity over vector addition: $r\cdot(\vec{v} + \vec{w}) = r\cdot\vec{v} + r\cdot\vec{w}$.
		\end{itemize} 

		\item \textbf{Inverse elements}:
		\begin{itemize}
			\item Additive inverse: $\forall \vec{v}\in V, \exists -\vec{v}\in V: \vec{v} + (-\vec{v}) = \vec{0}$.	
		\end{itemize} 

		\item \textbf{Identity elements}:
		\begin{itemize}
			\item Additive identity: $\exists\vec{0} \in V: \vec{0} + \vec{v} = \vec{v}, \quad \forall \vec{v}\in V$.
			\item Multiplicative identity: $\exists \bm{1}\in\field: \bm{1}\cdot\vec{v} = \vec{v},\quad \forall \vec{v} \in V$.	
		\end{itemize} 
	\end{enumerate} 

	\noindent For brevity, we will denote vectors as bold face letters instead of overhead arrows after this definition. For example, $\vu, \vv$ and $\vw$
\end{definition} 

\begin{remark}[Trivial Space]
	A vector space with one element is called a \textbf{trivial space}.	
\end{remark} 

\begin{example}
	The following is a vector space over $\R^2$:
	\begin{align*}
		\mathrm{L} = \bigCurl{
			\begin{pmatrix}
				x & y
			\end{pmatrix}^\top: y = 3x 
		}.
	\end{align*}

	\noindent This is easy to verify. Let us go through each condition one by one. Suppose that we have two vectors $\vu_1, \vu_2 \in \mathrm{L}$ defined as follows: 
\end{example} 