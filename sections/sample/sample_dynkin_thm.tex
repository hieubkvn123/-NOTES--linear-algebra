\section{Dynkin's $\pi$-$\lambda$ Theorem}
\noindent Before diving into the theorem, we should familiarise ourselves with the relevant definitions. Specifically, what is a $\pi$-system and what is a $\lambda$-system.

\subsection{$\pi$-system and $\lambda$-system}
\begin{definition}[$\pi$-system]
    Given a set $X$. A collection $\mathcal{P}$ of subsets of $X$ is called a $\pi$-system if it is \bf{closed under intersection}.
\end{definition}

\noindent The simplest example of a $\pi$-system is the set of any single elements of $X$ or the set that contains only the empty set. However, we are more interested in some of the more non-trivial examples of $\pi$-system:

\begin{itemize}
    \item The set of half-open intervals (from the left) : $\{(-\infty, a] : a \in \mathbb{R} \}$.
    \item The set of half-open intervals (from the right) : $\{ [a, \infty) : a \in \mathbb{R} \}$.
    \item The set of closed intervals are also a $\pi$-system if the empty set is included : $\{[a, b] : a, b \in \mathbb{R}; a\le b \} \cup \{ \emptyset \}$.
    \item If $\mathcal{P}_1, \mathcal{P}_1$ are $\pi$-systems over $X_1, X_2$ then the Cartesian products $\mathcal{P}_1\times \mathcal{P}_2 = \{ A_1 \times A_2 : A_1 \in \mathcal{P}_1, A_2 \in \mathcal{P}_2 \}$ is also a $\pi$-system over $X_1 \times X_2$.
    \item Any $\sigma$-algebra is a $\pi$-system.
\end{itemize}

\begin{definition}[$\lambda$-system]
    Given a set $X$. A collection of $\mathcal{D}$ of subsets of $X$ is called a $\lambda$-system if it satisfies the following conditions:
    \begin{itemize}
        \item $X \in \mathcal{D}$
        \item \textbf{Closure under relative complement} : If $A, B \in\mathcal{D}$ and $A \subseteq B \implies B \setminus A \in \mathcal{D}$.
        \item \textbf{Closure under countable disjoint union} : If there exists a countable collection of disjoint sets $\{A_n\}_{n=1}^\infty \subset \mathcal{D}$. Then, $\bigcup_{n=1}^\infty A_n \in \mathcal{D}$.
    \end{itemize}
\end{definition}

\noindent Now that we see that $\lambda$-system is actually a slightly more complicated algebraic structure than a $\pi$-system. However, one thing we can notice is that any $\sigma$-algebra is also a $\lambda$-system. More generally, we have the following proposition.

\begin{proposition}{$\sigma$-algebra = $\pi$-system + $\lambda$-system}{sigma_is_lambda_and_pi}
    Every $\sigma$-algebra is both a $\pi$-system and a $\lambda$-system.
\end{proposition}

\begin{proof*}[Proposition \ref{prop:sigma_is_lambda_and_pi}]
Given a set $X$ and let $\mathcal{A}$ be a $\sigma$-algebra on $X$. \newline 
\begin{subproof}{\textbf{Claim 1} : $\mathcal{A}$ is a $\pi$-system over $X$}
    We have to prove that $\mathcal{A}$ is closed under (finite) intersection. We know that $\mathcal{A}$ is closed under countable intersection. Hence, for all countable collection of sets $\{A_n\}_{n=1}^\infty$ in $\mathcal{A}$, we have:
    \begin{align*}
        &A = \bigcap_{n=1}^\infty A_n \in \mathcal{A}
    \end{align*}

    \noindent For all $N \ge 1$, we have:
    \begin{align*}
        \bigcap_{n=1}^N A_n &= A \setminus \bigcap_{m=N+1}^\infty A_m \\
            &= A \cap \Bigg( \bigcup_{m=N+1}^{\infty} A_m^c \Bigg) = \bigcup_{m=N+1}^\infty (A \cap A_m^c)
    \end{align*}

    \noindent For all $m \ge N+1$, $A\cap A_m^c$ is a countable intersection of sets in $\mathcal{A}$. Hence, $\bigcup_{m=N+1}^\infty (A \cap A_m^c)$ is a countable union of sets in $\mathcal{A}$. Hence, $\bigcap_{n=1}^N A_n \in \mathcal{A}$.

    \noindent Therefore, $\mathcal{A}$ is closed under finite intersection and is a $\pi$-system.\newline
\end{subproof}


\begin{subproof}{\textbf{Claim 2} : $\mathcal{A}$ is a $\lambda$-system over $X$}
    We have to prove that:
    \begin{itemize}
        \item $X \in \mathcal{A}$ : Trivial.
        \item $\mathcal{A}$ is closed under relative complement : For $A, B \in \mathcal{A}$ and $A\subseteq B$, we have $B\setminus A = B \cap A^c \in \mathcal{A}$ (By closure under intersection).
        \item $\mathcal{A}$ is closed under countable disjoint union : Trivial since $\mathcal{A}$ is already closed under countable union.
    \end{itemize}

    \noindent Hence, $\mathcal{A}$ is a $\lambda$-system over $X$.
\end{subproof}
\end{proof*}

\subsection{Theorem and proof}
\begin{theorem}{Dynkin's $\pi$-$\lambda$ Theorem}{dynkin}
    If $\mathcal{D}$ is a $\lambda$-system containing the $\pi$-system $\mathcal{P}$. Then, it also contains the $\sigma$-algebra generated by $\mathcal{P}$. 
    \begin{align*}
        \mathcal{P} \subseteq \mathcal{D} \implies \sigma(\mathcal{P}) \subseteq \mathcal{D}
    \end{align*}

    In other words, if a $\lambda$-system contains a $\pi$-system, it also contains the smallest $\sigma$-algebra containing the $\pi$-system.
\end{theorem}

\begin{proof*}[Theorem \ref{thm:dynkin}]
    We will prove the theorem by using the smallest $\lambda$-system generated from $\mathcal{P}$. We will prove that:
    \begin{itemize}
        \item $(i)$ $\lambda(\mathcal{P})$ is a $\sigma$-algebra.
        \item $(ii)$ $\sigma(\mathcal{P}) \subseteq \lambda(\mathcal{P})$.
        \item $(iii)$ Since $\lambda(\mathcal{P})\subseteq \mathcal{D} \implies \sigma(\mathcal{P}) \subseteq \lambda(\mathcal{P})\subseteq\mathcal{D}$.
    \end{itemize}

    \noindent Obviously, we know that $\lambda(\mathcal{P})$ is a $\lambda$-system, we have to prove that it is also a $\pi$-system. Meaning, $\forall A, B \in \lambda(\mathcal{P})\implies A\cap B \in \lambda(\mathcal{P})$. Let $A \in \lambda(\mathcal{P})$ be an arbitrary set and define:

    \begin{align*}
        \mathcal{L}_A = \Big\{ E : A\cap E \in \lambda(\mathcal{P}) \Big\}     
    \end{align*}

    \begin{subproof}{Claim 1 : $\mathcal{L}_A$ is a $\lambda$-system}
        We have:
        \begin{itemize}
    	\item $X\in \mathcal{L}_A$ because $A\cap X = A \in \lambda(\mathcal{P})$.
    	\item $\forall P, Q \in \mathcal{L}_A, P\subseteq Q \implies Q-P\in\mathcal{L}_A$ because we have:
        	\begin{itemize}
        	    \item $A\cap(Q-P) = (A\cap Q) - (A\cap P)$.
    		  \item $(A\cap Q), (A \cap P) \in \lambda(\mathcal{P})$  and we have $A\cap P \subseteq A\cap Q$. Hence, $(A\cap Q) - (A\cap P)\in\lambda(\mathcal{P})$.
            \end{itemize}
    	\item  $\forall \{E_n\}_{n=1}^\infty \subseteq \mathcal{L}_A$ be a disjoint collection, we have $\bigcup_{n=1}^\infty E_n \in \mathcal{L}_A$ because:
            \begin{itemize}
                \item $A\cap\bigcup_{n=1}^\infty E_n=\bigcup_{n=1}^\infty\{A\cap E_n\}$.
                \item For all $E_n$, $A\cap E_n \in \lambda(\mathcal{P})$ and disjoint, hence $\bigcup_{n=1}^\infty \{A\cap E_n\} \in \mathcal{L}_A$.\newline
            \end{itemize}
        \end{itemize}
    \end{subproof}


    \begin{subproof}{Claim 2 : $A\in\mathcal{P}\implies \lambda(\mathcal{P}) \subseteq \mathcal{L}_A$}
        We have:
        \begin{itemize}
    	\item $\forall C \in \mathcal{P}: A\cap C \in \mathcal{P}$ because both $A, C\in\mathcal{P}$.
    		\begin{itemize}[label={}]
                \item $\implies A\cap C \in \lambda(\mathcal{P})$.
        		\item $\implies \forall C\in\mathcal{P} : C\in\mathcal{L}_A$.
        		\item $\implies \mathcal{P}\subseteq \mathcal{L}_A$ (Meaning $\mathcal{L}_A$ is a $\lambda$-system generated  by $\mathcal{P})$.
            \end{itemize}
    	\item But, we already stated that $\lambda(\mathcal{P})$ is the smallest $\lambda$-system generated by $\mathcal{P}$. Hence, we have $\lambda(\mathcal{P})\subseteq\mathcal{L}_A$.\newline
        \end{itemize}
    \end{subproof}

    \begin{subproof}{Claim 3 : $B\in\lambda(\mathcal{P}) \implies \lambda(\mathcal{P}) \subseteq\mathcal{L}_B$}
        \begin{itemize}
            \item We already proved that $\forall A\in\mathcal{P}: \lambda(\mathcal{P})\subseteq \mathcal{L}_A$.
        	\item Hence, for an arbitrary $B\in \lambda(\mathcal{P}) \implies  B \in \mathcal{L}_A$.
        	\item In other words : $\forall A\in \mathcal{P}: A\cap B \in \mathcal{\lambda(\mathcal{P})} \implies \forall A \in \mathcal{P}: A \in \mathcal{L}_B$.
        		\begin{itemize}[label={}]
            		\item $\implies \mathcal{P}\subseteq \mathcal{L}_B$.
            		\item $\implies \lambda(\mathcal{P}) \subseteq \mathcal{L}_B$ (Again, $\lambda(\mathcal{P})$ is the smallest $\lambda$-system containing $\mathcal{P}$).\newline
                \end{itemize}
        \end{itemize}
    \end{subproof}

\noindent From \textbf{Claim 3}, we can conclude that for two arbitrary sets $A, B \in \lambda(\mathcal{P}) \implies A \in \mathcal{L}_B$ (and vice versa). Therefore, $A\cap B \in \lambda(\mathcal{P})$ and $\lambda(\mathcal{P})$ is also a $\pi$-system. Finally, we conclude that $\lambda(\mathcal{P})$ is a $\sigma$-algebra $_{\ \ \ \square}$.

\noindent\newline $(ii)$ We have proven that $\lambda(\mathcal{P})$ is a $\sigma$-algebra. We also have that $\sigma(\mathcal{P})$ is the smallest $\sigma$-algebra generated by $\mathcal{P}$. Hence, $\sigma(\mathcal{P})\subseteq\lambda(\mathcal{P})_{\ \ \ \square}$.

\noindent\newline $(iii)$ Since $\lambda(\mathcal{P})$ is the smallest $\lambda$-system containing $\mathcal{P}$. We have $\lambda(\mathcal{P})\subseteq \mathcal{D}$. Finally, we have $\sigma(\mathcal{P})\subseteq\lambda(\mathcal{P})\subseteq{D}$.
\end{proof*}

